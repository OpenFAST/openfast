\documentclass[11pt]{article}
% --------------------------------------------------------------------------------}
% --- Preamble
% --------------------------------------------------------------------------------{
\usepackage{ifpdf}
\ifpdf
	\usepackage{graphicx}
	\usepackage{epstopdf} %don't forget the shell-escape command for pdflatex
    \usepackage[colorlinks,bookmarksopen,bookmarksnumbered,citecolor=red,urlcolor=red]{hyperref}
\else
	\usepackage{graphicx}
    \usepackage[dvips,colorlinks,bookmarksopen,bookmarksnumbered,citecolor=red,urlcolor=red]{hyperref}
\fi
\usepackage{type1cm}
\usepackage{amsmath}
\usepackage{amssymb}
\usepackage{amsthm}
\usepackage{stmaryrd} %[| |] \llbracket
\usepackage{mhsetup}
\usepackage{mathtools}
\usepackage{relsize}
\usepackage{color} 
\usepackage[table]{xcolor}   %lots of tools for colors
\usepackage{import}
\usepackage{listings} % for  lstlistings
\usepackage{fullpage}
\usepackage{pdflscape}

\hypersetup{
    colorlinks = false,
    linkcolor =[rgb]{0.2,0.2,0.2},
    linkbordercolor =[rgb]{0.7,0.7,0.7}
}

\setlength{\itemsep}{-10em}
\DeclareGraphicsExtensions{.pdf,.PDF,.png,.jpg,.PNG,.JPG,.eps,.EPS,.Eps}
\graphicspath{{figs/}{figs_svg/}{figs_raw/}}
% --- svgtex
\newcommand{\svgtex}[4]{\begin{figure}[!htb]%
 \centering%
 \def\svgwidth{#3\columnwidth}% used to be a scalebox
 \scalebox{#4}{\import{figs_svgtex/}{#1}}%
 \caption{#2}\label{fig:#1}%
 \end{figure}%
}
% --- svg
\newcommand{\svg}[3]{\begin{figure}[!htb]
 \centering%
 \includegraphics[width=#3\textwidth]{#1}
 \caption{#2}\label{fig:#1}
 \end{figure}
}

\renewcommand{\v}[1]{\boldsymbol{#1}}
\newcommand{\m}[1]{\boldsymbol{#1}}
\newcommand{\eqdef}{\stackrel{\mathsmaller{\mathsmaller{\mathsmaller{\triangle}}}}{=}} 
\newcommand{\weird}[1]{{\color{red}{!!!#1!!!}}}


% ---- Commands specific to this project
\newcommand{\sh}{{S}}
\newcommand{\cent}{{C}}
\newcommand{\xsh}{x_\text{sh}}
\newcommand{\ysh}{y_\text{sh}}
\newcommand{\zsh}{z_\text{sh}}
\newcommand{\EA}{{E\!A}\,}
\newcommand{\EI}{E\!I}
\newcommand{\GK}{G\!K_t}
\newcommand{\GA}{G\!A}
\newcommand{\rhotil}{\m{\tilde{\rho}}}
\newcommand{\stil}  {\m{\tilde{s}}}

% --------------------------------------------------------------------------------}
% --- Document 
% --------------------------------------------------------------------------------{
\begin{document}
\title{BeamDyn inputs from sectional beam properties}
\author{E. Branlard, J. Jonkman}
\maketitle


\section{Coordinate system}
The coordinate system assumed for BeamDyn in this document is illustrated in \autoref{fig:BeamDynSectionCoord}. 
This coordinate system is similar to the one used by OpenFAST, with the exception that in this document angles are assumed positive around $z$, while in OpenFAST the angles are negative around $z$. As a result of this, all angles $\theta$ used in this document, need to be multiplied by $-1$ for applications with OpenFAST or BeamDyn. The OpenFAST coordinate system together with other aero-elastic codes coordinate systems are shown in  \autoref{fig:AeroElastCodesCoordConvention}.
% 
% --------------------------------------------------------------------------------
\svgtex{BeamDynSectionCoord}{Coordinate system assumed in this document for BeamDyn. Coordinates $x_b$,$y_b$,$z_b$ are the coordinates at the root of the blade, used to define the reference axis of the beam. Coordinates $x,y,z$ are used for a typical cross section, with $z$ normal to the cross section, tangent to the reference axis. Point $O$ is the origin of the cross section (defined by the reference axis), point $C$ the centroid, point $S$ the shear center and point $G$ the center of gravity. The coordinates of each points are expressed with respect to the origin of the cross section.}{1.0}{0.90}
% --------------------------------------------------------------------------------
The blade coordinate system includes rigid-body pitch rotation relative to the coned coordinate system.
The position/orientation, velocity, and acceleration of the blade coordinate system are passed from ElastoDyn to BeamDyn at every time step.


\section{Stiffness matrix}
The results presented in this section are inspired by the expressions given in the \textit{Dymore} manual~\cite{dymore} but for a different coordinate system\footnote{Note that the signs in equation (11) of \cite{dymore} are erroneous and have to be reversed. Apart from that, the coordinate mapping from this reference to the current document is simply $1\to z$, $2\to x$, $3\to y$.}.

\subsection{Introduction}
The stiffness matrix that relates the cross-section strains to the loads is assumed to take the following form (valid for orthotropic layups, that is, with no off-axis plies):
\begin{align}
    \begin{bmatrix}
    F_x \\ F_y\\ F_z \\ M_x \\ M_y\\ M_z\\ 
    \end{bmatrix}
&=
    \begin{bmatrix}
K_{11} & K_{12} & 0      & 0      & 0      & K_{16} \\
K_{21} & K_{22} & 0      & 0      & 0      & K_{26} \\
0      & 0      & K_{33} & K_{34} & K_{35} & 0      \\
0      & 0      & K_{43} & K_{44} & K_{45} & 0      \\
0      & 0      & K_{53} & K_{54} & K_{55} & 0      \\
K_{61} & K_{62} & 0      & 0       & 0  & K_{66} \\
\end{bmatrix}
    \begin{bmatrix}
    \gamma_{x} \\ \gamma_{y} \\     \epsilon_z \\ \kappa_x \\ \kappa_y\\ \kappa_z \\
    \end{bmatrix}
     \quad
    \begin{matrix}
   \text{(flapwise shear)}\\
   \text{(edgewise shear)}\\
   \text{(axial strain)}\\
   \text{(edgewise bending curvature)}\\
   \text{(flapwise bending curvature)}\\
   \text{(torsion)}\\
    \end{matrix}
\end{align}
where the text in parenthesis follows the convention used in \autoref{fig:BeamDynSectionCoord} to draw the airfoil.
When the stiffness matrix takes such form, the elements of the matrix involved in the axial and bending loads ($F_z,M_x,M_y$) and the elements of the matrix involved in the shear-torsion loads ($F_x,F_y,M_z$) may be determined independently. The two following sections consider these two independent problems. The elements of the matrix for both problems are expressed by considering the centroid and the principal axes of bending for the first problem, and the shear center and the principal shear directions for the second problem.



% \section*{Introduction}
% 
% \tableofcontents 
% \clearpage
\subsection{Axial force and bending moments}
The axial force is decoupled from the bending moments at the centroid (also referred to as the neutral axis or elastic axis) of the cross section. 
Futher, the bending moments are decoupled when expressed about the principal axes.
In this paragraph, the axial forces and bending moments are first expressed with respect to the centroid and the principal axes, then they are rotated to the cross section axis, and last translated to the origin of the cross section.
% 
The coordinates of the centroid, expressed from the origin of the cross sections are written $(x_\cent,y_\cent)$.
The principal axes, noted $\hat{x}_p,\hat{y}_p$, are rotated compared to the cross section axis with an angle $\theta_p$.
% 
The axial loads and bending moments about the centroid, and with respect to the principal axes of the cross section are:
\begin{align}
    \begin{bmatrix}F_z^\cent \\ M_{x_p}^\cent \\ M_{y_p}^\cent\\ \end{bmatrix}
    =
    \begin{bmatrix}\EA & 0 & 0\\0 & \EI_{x_p} & 0\\0 & 0& \EI_{y_p}\end{bmatrix}
    \begin{bmatrix}\epsilon_z^\cent \\ \kappa_{x_p}^\cent \\ \kappa_{y_p}^\cent\\ \end{bmatrix}
\end{align}
where the superscript $\cent$ indicates that the loads, stress and curvature are expressed at the centroid and the subscript $p$ indicates that the values are expressed in the frame of the princpal axes.
The transformation between the principal axes and the cross-section axis is such that:
\begin{align}
    \begin{bmatrix}F_z^\cent \\ M_{x}^\cent \\ M_{y}^\cent\\ \end{bmatrix}
    &=
        \m{T}
    \begin{bmatrix}\EA & 0 & 0\\0 & \EI_{x_p} & 0\\0 & 0& \EI_{y_p}\end{bmatrix}
        \m{T}^T
    \begin{bmatrix}\epsilon_z^\cent \\ \kappa_x^\cent \\ \kappa_y^\cent\\ \end{bmatrix}
        =
    \begin{bmatrix}\EA & 0 & 0\\0 & H_{xx} & - H_{xy}\\0 & - H_{xy} & H_{yy}\end{bmatrix}
    \begin{bmatrix}\epsilon_z^\cent \\ \kappa_x^\cent \\ \kappa_y^\cent\\ \end{bmatrix}
            ,
    \label{eq:AxialBendingAtCentroid}
        \\
\shortintertext{with}
    \m{T}&=\begin{bmatrix}1 & 0 & 0\\0 & \cos\theta_p & -\sin\theta_p\\0 & \sin\theta_p& \cos\theta_p\end{bmatrix}
\end{align}
and where the variables $H_{xx},H_{xy},H_{yy}$ represent the bending stiffnesses in the cross-section coordinates system, obtained from the principal axis stiffnesses as:
\begin{align}
H_{xx} &=  \EI_{x_p} \cos^{2}\theta_p + \EI_{y_p} \sin^{2}\theta_p \\
H_{yy} &= \EI_{x_p} \sin^{2}\theta_p + \EI_{y_p} \cos^{2}{\theta_p}  \\
H_{xy} &= -\EI_{x_p} \sin\theta_p \cos\theta_p + \EI_{y_p} \sin\theta_p \cos\theta_p
\end{align}
% 
% 
% The axial loads and the bending moments are decoupled , that is:
% \begin{align}
%     \begin{bmatrix}F_z^\cent \\ M_x^\cent \\ M_y^\cent\\ \end{bmatrix}=
%     \begin{bmatrix}\EA & 0 & 0\\0 & H_{xx} & - H_{xy}\\0 & - H_{xy} & H_{yy}\end{bmatrix}
%     \begin{bmatrix}\epsilon_z^\cent \\ \kappa_x^\cent \\ \kappa_y^\cent\\ \end{bmatrix}
%     \label{eq:AxialBendingAtCentroid}
% \end{align}
    
%   
% 
%    
%   
% 
% 
The transformation of the loads, axial stress and curvatures between the origin of the section and the centroid are such that:
\begin{align}
    \begin{bmatrix}F_z \\ M_x \\ M_y\\ \end{bmatrix}
    =
    \begin{bmatrix}1 & 0 & 0\\y_{\cent} & 1 & 0\\- x_{\cent} & 0 & 1\end{bmatrix}
    \begin{bmatrix}F_z^\cent \\ M_x^\cent \\ M_y^\cent\\ \end{bmatrix}
    %\label{eq:}
    ,\qquad
    \begin{bmatrix}\epsilon_z^\cent \\ \kappa_x^\cent \\ \kappa_y^\cent\\ \end{bmatrix}=
    \begin{bmatrix}1 & y_{\cent} & - x_{\cent}\\0 & 1 & 0\\0 & 0 & 1\end{bmatrix}
    \begin{bmatrix}\epsilon_z \\ \kappa_x \\ \kappa_y\\ \end{bmatrix}
    \label{eq:AxialBendingCoordTrans}
\end{align}
% 
% 
Using \autoref{fig:StiffnessMatrixAxialBending}, the signs can be verified as follows: a positive axial force at point $\cent$ ($F_z^\cent>0$), leads to a negative moment about $y$ and a positive moment about $x$ at the origin; a positive $x$-curvature of the beam at the origin ($\kappa_x>0$), leads to an elongation of the fibers at $\cent$, whereas a positive $y$-curvature of the beam at the origin implies a compression of the fibers at point $\cent$ ($\epsilon_z^\cent<0)$
% 
% --------------------------------------------------------------------------------
\svgtex{StiffnessMatrixAxialBending}{Strain and loads transformation between the centroid $C$ and the origin $O$.}{1.1}{0.85}
% --------------------------------------------------------------------------------
% 
% 
Combining \autoref{eq:AxialBendingCoordTrans} and \autoref{eq:AxialBendingAtCentroid} leads to:
\begin{align}
    \begin{bmatrix}F_z \\ M_x \\ M_y\\ \end{bmatrix}
    =
    \begin{bmatrix}
    \EA & \EA y_{\cent} & - \EA x_{\cent}\\
    \EA y_{\cent} & H_{xx} + \EA y_{\cent}^{2} & - H_{xy} - \EA x_{\cent} y_{\cent}\\
  - \EA x_{\cent} & - H_{xy} - \EA x_{\cent} y_{\cent} & H_{yy} + \EA x_{\cent}^{2}\end{bmatrix}
    \begin{bmatrix}\epsilon_z \\ \kappa_x \\ \kappa_y\\ \end{bmatrix}
    \label{eq:AxialBendingOrigin}
\end{align}



% --------------------------------------------------------------------------------}
% ---  
% --------------------------------------------------------------------------------{
\subsection{Twisting moment and shear-forces}
The torsional moment and the shear forces are decoupled at the shear-center of the cross section, noted $\sh$, and of coordinates $(x_\sh, y_\sh)$ with respect to the origin of the cross section.
Further, the shear forces are independent when expressed with respect to the principal shear directions (usually taken as the principal axes direction), which are the axes, noted $\hat{x}_s, \hat{y}_s$, obtained by rotating the cross section axes $\hat{x}$, $\hat{y}$ by an angle $\theta_s$ about the $z$-axis.
The stiffness matrix about the shear center and in the principal shear directions is then:
\begin{align}
    \begin{bmatrix} F_{x_s}^\sh \\ F_{y_s}^\sh\\M_z^\sh \\ \end{bmatrix}
    =
    \begin{bmatrix}
     k_x \GA & 0 & 0\\
     0 & k_y \GA &0  \\
     0 & 0  & \GK\\
    \end{bmatrix}
    \begin{bmatrix}\gamma_{x_s}^\sh \\ \gamma_{y_s}^\sh \\ \kappa_{z}^\sh \end{bmatrix}
    \label{eq:ShearTorsionAtShearCenterInShearDirections}
\end{align}
where the superscript $\sh$ indicates that the quantities are expressed at the shear center while the subscript $s$ indicates that the values are related to the principal shear directions. The variables $k_x$ and $k_y$ are the dimensionless shear factor related to shear forces in the $\hat{x}_s$ and $\hat{y}_s$ direction respectively.
    The stiffness matrix at the shear center is transformed to the cross-section frame leading to:
\begin{align}
    \begin{bmatrix} F_x^\sh \\ F_y^\sh\\M_z^\sh \\ \end{bmatrix}
    &=
    \begin{bmatrix}
     K_{xx} & - K_{xy} & 0\\
     - K_{xy} & K_{yy} &0  \\
     0 & 0  & \GK\\
    \end{bmatrix}
    \begin{bmatrix}\gamma_{x}^\sh \\ \gamma_{y}^\sh \\ \kappa_z^\sh \end{bmatrix}
    \label{eq:ShearTorsionAtShearCenter}
    \shortintertext{with}
K_{xx}/\GA &=   k_{x_s} \cos^{2}\theta_s +  k_{y_s} \sin^{2}\theta_s \\
K_{yy}/\GA &= k_{x_s} \sin^{2}\theta_s + k_{y_s} \cos^{2}{\theta_s} \\
K_{xy}/\GA &= (k_{y_s}-k_{x_s}) \sin\theta_s \cos\theta_s
% 
\end{align}
% The shear center has the coordinates $(x_\sh,y_\sh)$ with respect to the origin of the cross section.
The loads, strains and twisting rate are transferred from the origin to the shear center as follows:
% 
\begin{align}
    \begin{bmatrix}F_x \\ F_y\\ M_z\\ \end{bmatrix}
    =
    \begin{bmatrix}
     1 & 0 &0 \\
         0 & 1 & 0 \\
    - y_{\sh} & x_{\sh}& 1\\
    \end{bmatrix}
    \begin{bmatrix} F_x^\sh \\ F_y^\sh\\M_z^\sh \\ \end{bmatrix}
,\qquad
    \begin{bmatrix}\gamma_{x}^\sh \\ \gamma_{y}^\sh \\ \kappa_z^\sh \end{bmatrix}
    =
    \begin{bmatrix}1 & 0 & - y_{\sh}\\0 & 1 & x_{\sh}\\0 & 0 & 1\end{bmatrix}
    \begin{bmatrix}\gamma_{x} \\ \gamma_{y} \\ \kappa_z \end{bmatrix}
    \label{eq:ShearTorsionCoordTrans}
\end{align}
% The signs can be verified as follows:
% 
Combining \autoref{eq:ShearTorsionAtShearCenter} with  \autoref{eq:ShearTorsionCoordTrans} leads to:
% 
\begin{align}
    \begin{bmatrix}F_x \\ F_y\\ M_z\\ \end{bmatrix}
    &=
    \begin{bmatrix}
K_{xx} & - K_{xy} & - K_{xx} y_{\sh} - K_{xy} x_{\sh}\\- K_{xy} & K_{yy} & K_{xy} y_{\sh} + K_{yy} x_{\sh}\\- K_{xx} y_{\sh} - K_{xy} x_{\sh} & K_{xy} y_{\sh} + K_{yy} x_{\sh} & \GK + K_{xx} y_{\sh}^{2} + 2 K_{xy} x_{\sh} y_{\sh} + K_{yy} x_{\sh}^{2}\end{bmatrix}
    \begin{bmatrix}\gamma_{x} \\ \gamma_{y} \\ \kappa_z \end{bmatrix}
    \label{eq:ShearTorsionOrigin}
\end{align}





\subsection{Summary}
The results given in \autoref{eq:ShearTorsionOrigin} and \autoref{eq:AxialBendingOrigin} are combined below to form the $6\times 6$ stiffness matrix expressed at the origin of the cross section:
\begin{align}
    \begin{bmatrix}
    F_x \\ F_y\\ F_z \\ M_x \\ M_y\\ M_z\\ 
    \end{bmatrix}
&=
    \begin{bmatrix}
K_{11} & K_{12} & 0      & 0      & 0      & K_{16} \\
K_{21} & K_{22} & 0      & 0      & 0      & K_{26} \\
0      & 0      & K_{33} & K_{34} & K_{35} & 0      \\
0      & 0      & K_{43} & K_{44} & K_{45} & 0      \\
0      & 0      & K_{53} & K_{54} & K_{55} & 0      \\
K_{61} & K_{62} & 0      & 0       & 0  & K_{66} \\
\end{bmatrix}
% 
    \begin{bmatrix}
    \gamma_{x} \\ \gamma_{y} \\     \epsilon_z \\ \kappa_x \\ \kappa_y\\ \kappa_z \\
    \end{bmatrix}
    \\
    \begin{bmatrix}
K_{11} &  K_{12} & K_{16} \\
K_{21} &  K_{22} & K_{26} \\
K_{61} &  K_{62} & K_{66} \\
\end{bmatrix}
&=
    \begin{bmatrix}
K_{xx} & - K_{xy} & - K_{xx} y_{\sh} - K_{xy} x_{\sh}\\- K_{xy} & K_{yy} & K_{xy} y_{\sh} + K_{yy} x_{\sh}\\- K_{xx} y_{\sh} - K_{xy} x_{\sh} & K_{xy} y_{\sh} + K_{yy} x_{\sh} & \GK + K_{xx} y_{\sh}^{2} + 2 K_{xy} x_{\sh} y_{\sh} + K_{yy} x_{\sh}^{2}
\end{bmatrix}
\nonumber
\\
    \begin{bmatrix}
K_{33} &  K_{34} & K_{35} \\
K_{43} &  K_{44} & K_{45} \\
K_{53} &  K_{54} & K_{55} \\
\end{bmatrix}
&=
    \begin{bmatrix}\EA & \EA y_{\cent} & - \EA x_{\cent}\\\EA y_{\cent} & H_{xx} + \EA y_{\cent}^{2} & - H_{xy} - \EA x_{\cent} y_{\cent}\\- \EA x_{\cent} & - H_{xy} - \EA x_{\cent} y_{\cent} & H_{yy} + \EA x_{\cent}^{2}\end{bmatrix}
\nonumber
    \\
H_{xx} &=  \EI_{x_p} \cos^{2}\theta_p + \EI_{y_p} \sin^{2}\theta_p \nonumber \\
H_{yy} &= \EI_{x_p} \sin^{2}\theta_p + \EI_{y_p} \cos^{2}{\theta_p} \nonumber \\
H_{xy} &= (\EI_{y_p}-\EI_{x_p}) \sin\theta_p \cos\theta_p  \nonumber \\
K_{xx}/\GA &=   k_{x_s} \cos^{2}\theta_s +  k_{y_s} \sin^{2}\theta_s \nonumber\\
K_{yy}/\GA &= k_{x_s} \sin^{2}\theta_s + k_{y_s} \cos^{2}{\theta_s} \nonumber \\
K_{xy}/\GA &= (k_{y_s}-k_{x_s}) \sin\theta_s \cos\theta_s \nonumber
\end{align}





\clearpage
% --------------------------------------------------------------------------------}
% ---  
% --------------------------------------------------------------------------------{
\section{Mass matrix}

\subsection{General form for a rigid body}
The mass matrix of a rigid body expressed at its center of mass is 
\begin{align}
\m{M}^G=
    \begin{bmatrix}
        M\m{I}_3 & 0  \\
        0  &  \m{J}^{G}  \\
    \end{bmatrix}
\end{align}
The general form of the mass matrix of a rigid body expressed at a given point $O$ is:
\begin{align}
\m{M}^O=
    \begin{bmatrix}
        M\m{I}_3 & -M \rhotil  \\
        M \rhotil  &  \m{J}^{O}  \\
    \end{bmatrix}
\end{align}
where
$\v{\rho}\eqdef \v{r}_{{O}G}$ is the distance from point $O$ to point $G$ and 
$\m{J}^{O}$ is the inertia tensor of the body at $O$, related to the inertia tensor at the COG by:
% \begin{align}
$
    \m{J}^{O}\eqdef - \int \stil_{P}\, \stil_{P}\, dm 
=\m{J}_G-M \rhotil\rhotil
$
,  $\v{s}_P=\v{r}_{OP}$ is a point of the body
and  where the tilde notation refer to the \textit{skew symmetric matrix}.
Given two vectors $\v{u}$ and $\v{t}$, the skew symmetric matrix is such that $\tilde{\v{u}}\; \v{t}  =\v{u}\times\v{t}$.
which is written in matricial form as follows:
\begin{align}
%     \text{Skew}(\v{u}) \eqdef 
        \v{\tilde{u}} \eqdef 
    \begin{bmatrix}
        0    & -u_z & u_y \\
        u_z  & 0    & -u_x \\
        -u_y & u_x & 0 \\
    \end{bmatrix}  %\label{eq:}
\end{align}



\subsection{Mass matrix of a cross section}

Mass matrix about center of mass $G$, and about the principal inertia directions:
\begin{align}
\m{M}_i^G = 
  \begin{bmatrix}
  m & 0 & 0 & 0       & 0      &0 \\
  0 & m & 0 & 0       & 0      &0 \\
  0 & 0 & m & 0       & 0      &0 \\
  0 & 0 & 0 & I_{x_i} & 0      &0 \\
  0 & 0 & 0 & 0       & I_{y_i} &   \\
  0 & 0 & 0 & 0       & 0       &  I_p \\
  \end{bmatrix}
  ,\qquad
\end{align}
where $I_p=I_{x_i} + I_{y_i}$.
Rotated to be expressed about the cross section axis, this becomes:
\begin{align}
\m{M}^G &= 
  \begin{bmatrix}
  m & 0 & 0 & 0       & 0       & 0   \\
  0 & m & 0 & 0       & 0       & 0   \\
  0 & 0 & m & 0       & 0       & 0   \\
  0 & 0 & 0 & I_{xx}  & -I_{xy} & 0   \\
  0 & 0 & 0 & -I_{xy} & I_{yy}  &     \\
  0 & 0 & 0 & 0       & 0       & I_p \\
  \end{bmatrix}
  ,\qquad
  \shortintertext{with}
  I_{xx} &=  I_{x_i} \cos^2\theta_i + I_{y_i} \sin^2\theta_i\\
  I_{yy} &=  I_{x_i} \sin^2\theta_i + I_{y_i} \cos^2\theta_i\\
  I_{xy} &=  (I_{y_i}-I_{x_i}) \sin\theta_i \cos\theta_i
\end{align}
Transferred to the origin:
\begin{align}
\m{M}^O=
\begin{bmatrix}
m & 0 & 0 & 0 & 0 & - m y_G\\0 & m & 0 & 0 & 0 & m x_G\\0 & 0 & m & m y_G & - m x_G & 0\\0 & 0 & m y_G & I_{xx} + m y_G^{2} & - I_{xy} - m x_G y_G & 0\\0 & 0 & - m x_G & - I_{xy} - m x_G y_G & I_{yy} + m x_G^{2} & 0\\- m y_G & m x_G & 0 & 0 & 0 & I_{p} + m x_G^{2} + m y_G^{2}
\end{bmatrix}
\end{align}




% 
% 
% \subsection{Old}
% % \weird{TODO}
% Example of mass matrix , beam directed along $x$
% % See DeFriasLopez p 17
% \begin{align}
%   \begin{bmatrix}
%   m & 0     & 0      & 0      & 0      & 0      \\
%   0 & m     & 0      & z_G m  & 0      & 0      \\
%   0 & 0     & m      & -y_G m & 0      & 0      \\
%   0 & z_G m & -y_G m & \rho_t & 0      & 0      \\
%   0 & 0     & 0      & \rho_t & 0      & 0      \\
%   0 & 0     & 0      & 0      & \rho_y & 0      \\
%   0 & 0     & 0      & 0      & 0      & \rho_z \\
%   \end{bmatrix}
% \end{align}
% 
% 
% \weird{TODO}
% Example of mass matrix , beam directed along $x$
% % See FAST documentation
% %     https://openfast.readthedocs.io/en/master/source/user/beamdyn/input_files.html#equation-Stiffness
% \begin{align}
%   \begin{bmatrix}
%   m               & 0     & 0      & 0            & 0            & -y_G m       \\
%   0               & m     & 0      & 0            & 0            & x_G m        \\
%   0               & 0     & m      & y_G m        & -m x_G       & 0            \\
%   0               & 0     & y_G m  & \rho_t       & 0            & 0            \\
%   0               & 0     & -x_G m & i_\text{Edg} & -i_\text{cp} & 0            \\
%   0               & 0     & 0      & -i_\text{cp} & i_\text{Flp} & 0            \\
%   -y_G m          & m x_G & 0      & 0            & 0            & i_\text{plr} \\
%   \end{bmatrix}
%   ,\qquad
% \end{align}
% \weird{$i_\text{plr}=i_\text{Edg}+i_\text{Flp}$, when\ldots}
% 
% 
% \clearpage
% --------------------------------------------------------------------------------}
% ---  
% --------------------------------------------------------------------------------{
\section{Setting BeamDyn inputs from HAWC2 inputs}
To setup a BeamDyn model based on a HAWC2 model, it is convenient to use the mid-chord as a reference line of the beam. This is indeed the reference used in HAWC2, referred to as the ``c2\_def''-coordinate system.
% The definition of the mean line in BeamDyn and HAWC2 is given in \autoref{tab:MeanLineHAWC2}.
%
The structural file of HAWC2 contains the location of the center of gravity, shear center, and  centroid. The centroid is yet referred to as ``elastic-center'' (defined as the point where the axial force is decoupled from the bending around $x$ and $y$).
%
The bending stiffness properties of HAWC2 are defined with respect to the principal axes, rotated by an structural pitch angle $\theta_s$ around $z$, compared to the axis of the cross section. The principal axes and principal shear axis are assumed to be the same.
%
The correspondence between the notations of the current document and HAWC2 is given in \autoref{tab:MeanLineHawc2}.
% ----------------------------------- TABLE --------------------------------------
% \begin{landscape}
\begin{table}[!h]\centering
    \caption{Correspondence between the current definitions and the inputs from HAWC2 for the mean line, section coordinates and stiffness properties}
    \label{tab:MeanLineHawc2}
\begin{tabular}{rcl}
  \textbf{Current}    &   & \textbf{HAWC2}                \\
    \hline
(\tt{kp\_xr}) $x_O$     & = & $ y_{c2}$    (\tt{y-pos}) \\
(\tt{kp\_yr}) $y_O$     & = & $ -x_{c2}$   (``-''\tt{x-pos}) \\
(\tt{kp\_zr}) $z_O$     & = & $ z_{c2}$    (\tt{z-pos}) \\
(\tt{initial\_twist}) $\theta_{z}$ & = & $ \theta_{z}$(\tt{theta\_z}) \\
    \hline
 $x_G$   & = & $ y_{m}$  \\
 $y_G$   & = & $ -x_{m}$ \\
 $x_\sh$ & = & $ y_{s}$  \\
 $y_\sh$ & = & $ -x_{s}$ \\
 $x_\cent$ & = & $ y_{e}$  \\
 $y_\cent$ & = & $ -x_{e}$ \\
\hline
 $\EA$       & = & $ E\, A$   \\
 $\GK$       & = & $ G\, K$   \\
 $k_{x_s}$       & = & $ k_y  $   \\
 $k_{y_s}$       & = & $ k_x  $   \\
 $\EI_{x_p}$ & = & $ E I_y  $ \\
 $\EI_{y_p}$ & = & $ E I_x  $ \\
 $\theta_s$ & = & $ \theta_s$ \\
 $\theta_p$ & = & $ \theta_s$ \\
\hline
 $\theta_i$ & = & $ \theta_p$ \\
 $m$ &        = & $m$ \\
 $I_{x_i}$ &   = & $r_{iy}^2 m$ \\
 $I_{y_i}$ &   = & $r_{ix}^2 m$ \\
 $I_{p}$   &   = & $K  m / A$\\
\hline
 \end{tabular}
\end{table}
% \end{landscape}
% --------------------------------------------------------------------------------
% ----------------------------------- TABLE --------------------------------------
% % \begin{landscape}
% \begin{table}[!h]\centering
%     \caption{Correspondence of section coordinates between the current document and hawc2}
%     \label{tab:SecCoord}
% \begin{tabular}{rcl}
%   \textbf{Current}    &   & \textbf{Hawc2}                \\
%     \hline
%  $x_G$   & = & $ y_{m}$  \\
%  $y_G$   & = & $ -x_{m}$ \\
%  $x_\sh$ & = & $ y_{s}$  \\
%  $y_\sh$ & = & $ -x_{s}$ \\
%  $x_\cent$ & = & $ y_{e}$  \\
%  $y_\cent$ & = & $ -x_{e}$ \\
% \hline
%  \end{tabular}
% \end{table}
% % \end{landscape}
% --------------------------------------------------------------------------------
% ----------------------------------- TABLE --------------------------------------
% % \begin{landscape}
% \begin{table}[!h]\centering
%     \caption{Correspondence of stiffness properties between the current document and hawc2}
%     \label{tab:StiffHawc2}
% \begin{tabular}{rcl}
%   \textbf{Current}    &   & \textbf{Hawc2}                \\
%     \hline
%  $\EA$       & = & $ E\, A$   \\
%  $\GK$       & = & $ G\, K$   \\
%  $k_x$       & = & $ k_y  $   \\
%  $k_y$       & = & $ k_x  $   \\
%  $\EI_{x_p}$ & = & $ E I_y  $ \\
%  $\EI_{y_p}$ & = & $ E I_x  $ \\
%  $\theta_s$ & = & $ \theta_s$ \\
%  $\theta_p$ & = & $ \theta_s$ \\
% \hline
%  \end{tabular}
% \end{table}
% % \end{landscape}
% --------------------------------------------------------------------------------








% \clearpage
% 
% \section{6x6 matrices and element matrices}
% \weird{TODO}
% % See DeFriasLopez p 15
% \begin{align}
%     k_e = \int_{0}^L \m{B_k}^T  \m{C} \m{B_k} dx \\
%     m_e = \int_{0}^L \m{B_m}^T  \m{\rho} \m{B_m} dx \\
% \end{align}
% where $\m{B_k}$ and $\m{B_m}$ are usually taken as the Hermit or cubic polynomial matrices.
% 
% 
% 



\appendix



   
\clearpage
% --------------------------------------------------------------------------------}
% ---  
% --------------------------------------------------------------------------------{
\section{Typical coordinate systems for different aero-elastic codes}
The OpenFAST coordinate system together with other aero-elastic codes coordinate systems are shown in  \autoref{fig:AeroElastCodesCoordConvention}. The OpenFAST coordinate system follows the IEC wind turbine convention. Angles are negative about $z$.
% --------------------------------------------------------------------------------
\svgtex{AeroElastCodesCoordConvention}{Coordinate systems for three aeroelastic codes: OpenFAST, HAWC2 and Flex. The airfoil cross section is drawn for a typical bending and sweep of an upwind turbine.}{1.3}{0.75}
% --------------------------------------------------------------------------------
\clearpage






% 
% 
% \section{3D Beam representation - Timoshenko beam and 3D frames}
% 
% 
% 
% 
% 
% 
% % --------------------------------------------------------------------------------}
% % --- Frame 
% % --------------------------------------------------------------------------------{
% \subsection{3D frame element}
% The 3D frame element stiffness matrix  for an element directed along $x$ is directly obtained from the more general Timoshenko element with zero shear deformation parameters ($\Phi_y=\Phi_z=0$) and a shear center at zero ($\ysh=\zsh=0$) in \autoref{eq:KTimoshenkoOffsets}.
% If the frame is directed along $z$, the stiffness matrix from \autoref{eq:KTimoshenkoOffsetsAlongZ} can similarly be used.
% 
% % % --- Stiffness matrix
% % a =      EA  / L   ; % a1
% % b = 12 * EIz / L^3 ; % b1
% % c = 6  * EIz / L^2 ; % b2
% % d = 12 * EIy / L^3 ; % c1
% % e = 6  * EIy / L^2 ; % c2
% % f =     G*Kv / L   ; % a2 = G*J/L
% % g = 2  * EIy / L   ; % c3
% % h = 2  * EIz / L   ; % b3
% % 
% % % NOTE: OK with
% % %          - Serano beam3e function
% % %          - Matlab FEM Book
% % %          - frame3d_6j
% % %          - Panzer-Hubele
% % % NOTE: compatible with by Timoshenko with shear offsets
% % %    ux1 uy1 uz1 tx1 ty1 tz1   ux2  uy2 yz2 tx2 ty2 tz2
% % ke = [ a   0  0   0   0   0    -a   0   0   0   0   0  ;% ux1
% %       0   b   0   0   0   c     0  -b   0   0   0   c  ;% uy1
% %       0   0   d   0  -e   0     0   0  -d   0  -e   0  ;% uz1
% %       0   0   0   f   0   0     0   0   0  -f   0   0  ;% tx1
% %       0   0  -e   0  2*g  0     0   0   e   0   g   0  ;% ty1
% %       0   c   0   0   0  2*h    0  -c   0   0   0   h  ;% tz1
% % 
% %      -a   0   0   0   0   0     a   0   0   0   0   0  ;% ux2
% %       0  -b   0   0   0  -c     0   b   0   0   0  -c  ;% uy2
% %       0   0  -d   0   e   0     0   0   d   0   e   0  ;% yz2
% %       0   0   0  -f   0   0     0   0   0   f   0   0  ;% tx2
% %       0   0  -e   0   g   0     0   0   e   0  2*g  0  ;% ty2
% %       0   c   0   0   0   h     0  -c   0   0   0  2*h];% tz2
% 
% % % SOURCE: What-When-How-FEM-For-Frames. NOTE: the sign was reveresed in front of 35*r2!!!, to be consistent with Panzer-Hubele with Iy and Iz=0
% % a  = L/2 ; a2 = a^2 ; r2 = EIx/E/A;
% % me = Mass/2/105 * [
% % %ux1  uy1     uz1   tx1     ty1   tz1     ux2    uy2    yz2    tx2    ty2    tz2
% %  70     0     0       0     0      0      35     0      0       0      0      0    ;% ux1
% %   0    78     0       0     0   22*a       0    27      0       0      0  -13*a    ;% uy1
% %   0     0    78       0 -22*a      0       0     0     27       0   13*a      0    ;% uz1
% %   0     0     0   70*r2     0      0       0     0      0   35*r2      0      0    ;% tx1
% %   0     0 -22*a       0  8*a2      0       0     0  -13*a       0  -6*a2     0     ;% ty1
% %   0  22*a     0       0     0   8*a2       0  13*a      0       0      0  -6*a2    ;% tz1
% % % 
% %  35     0     0       0     0      0      70     0      0       0      0      0    ;% ux2
% %   0    27     0       0     0   13*a       0    78      0       0      0  -22*a    ;% uy2
% %   0     0    27       0 -13*a      0       0     0     78       0   22*a      0    ;% yz2
% %   0     0     0   35*r2     0      0       0     0      0   70*r2      0      0    ;% tx2
% %   0     0  13*a       0 -6*a2      0       0     0   22*a       0   8*a2      0    ;% ty2
% %   0 -13*a     0       0     0  -6*a2       0 -22*a      0       0      0   8*a2   ];% tz2
% 
% 
% 
% 
% 
% 
% 
% 
% % --------------------------------------------------------------------------------}
% % --- Timoshenko along x 
% % --------------------------------------------------------------------------------{
% \subsection{Timoshenko beam with shear center offsets - Beam along x}
% \label{sec:TimoshenkoBeamAlongX}
% 
% The formulation for a Timoshenko beam is found e.g. in the book of Przemieniecki~\cite[p.79]{przemieniecki:book} (see also \cite{panzer:2009}) for $\ysh=\zsh=0$.
% 
% % SEE ALSO:  - Generating a Parametric Finite Element Model of a 3D Cantilever Timoshenko Beam Using Matlab
% % NOTE: compatible with above
% The general formulation with shear offsets is given below.
% The shear-deformation parameters $\Phi_y$ and $\Phi_z$ are defined as: 
% \begin{align}
% \Phi_y 
% &= \frac{12 EI_z}{k_y A G L^2}
% = \frac{12 EI_z}{{A_{s_y}} G L^2}
% = 24 ( 1+\nu) \frac{A}{A_{s_y}}\left(\frac{r_z}{L}\right)^2 \nonumber\\
% \Phi_z 
% &= \frac{12 EI_y}{k_z A G L^2}
% = \frac{12 EI_y}{{A_{s_z}} G L^2}
% = 24 ( 1+\nu) \frac{A}{A_{s_z}}\left(\frac{r_y}{L}\right)^2
%  \label{eq:ShearDeformationParams}
% \end{align}
% where the last equality assumes an isotropic linear elastic material such that $G=E/(2\nu +1)$ with $\nu$ the Poisson's ratio,
% where $r_y$ and $r_z$ are the radii of gyration, 
% $A_{s_y}$ and $A_{s_y}$ are the effective area of shear (for a rectangular cross section they are both equal to $5/6 th$)
% $k_y$ and $k_z$ are the dimensionless shear factors for the force in the principal bending axis direction $y$ and $z$ respectively. The following notations are sometimes adopted by some authors: $\eta_y=\Phi_z/12$,  $\eta_z=\Phi_y/12$, $\rho_y=1/(1+\Phi_z)$, $\rho_z=1/(1+\Phi_y)$ (note the inversion between $y$ and $z$ in these definitions).
% % 
% The shear-deformation parameters $\Phi_y$ and $\Phi_z$ are $0$ for a Euler-Bernoulli beam.
% % 
% The stiffness matrix for a Timoshenko beam along the $x$ direction, with shear center offsets $\ysh$, $\zsh$ is given in \autoref{eq:KTimoshenkoOffsets}, where the following notation was introduced:
% \begin{align}
%     K_\text{sh} = \frac{12E}{L^3}\left[\frac{I_y \ysh^2}{1+\Phi_z}  + \frac{I_z \zsh^2}{1+\Phi_y} \right] \label{eq:Ksh}
% \end{align}
% and the notation $k_{ij}$ is used to denote the element of row $i$ and column $j$ of the matrix.
% 
% \renewcommand*{\arraystretch}{1.9}
% \begin{align}
% \setcounter{MaxMatrixCols}{20}
% % \m{K}&=\\
% &\text{Stiffness matrix for a Timoshenko beam element along $x$ with shear center offsets}\label{eq:KTimoshenkoOffsets}\\
% &\!\!  \begin{bmatrix}
%     \frac{EA}{L} &                                &                                &                              &                                    &                                    &  &  &  &  &  &  \\
%     0            & \frac{12 E I_z}{L^3(1+\Phi_y)} &                                &                              &                                    &                                    &  &  &  &  &  &  \\
%     0            & 0                              & \frac{12 E I_y}{L^3(1+\Phi_z)} &                              &                                    &    \text{sym.}                     &  &  &  &  &  &  \\
%     0            & -k_{22}\,\zsh                  & k_{33}\,\ysh                   & \frac{GI_x}{L} + K_\text{sh} &                                    &                                    &  &  &  &  &  &  \\
%     0            & 0                              & -\frac{6EI_y}{L^2(1+\Phi_y)}   & k_{53}\,\ysh                 & \frac{EI_y(4+\Phi_z)}{L(1+\Phi_z)} &                                    &  &  &  &  &  &  \\
%     0            & \frac{6 EI_z}{L^2(1+\Phi_y)}   & 0                              & -k_{62}\,\zsh                & 0                                  & \frac{EI_z(4+\Phi_y)}{L(1+\Phi_y)} &  &  &  &  &  &  \\
%     %
%     -k_{11} & 0             & 0             & 0             & 0                                  & 0                                  & k_{11} &         &         &         &        &        \\
%     0       & -k_{22}       & 0             & k_{22}\,\zsh  & 0                                  & -k_{62}                            & 0      & k_{22}  &         &         &        &        \\
%     0       & 0             & -k_{33}       & -k_{33}\,\ysh & -k_{53}                            & 0                                  & 0      & 0       & k_{33}  &         &        &        \\
%     0       & k_{22}\, \zsh & -k_{33}\,\ysh & -k_{44}       & -k_{53}\,\ysh                      & k_{62}\,\zsh                       & 0      & k_{42}  & k_{43}  & k_{44}  &        &        \\
%     0       & 0             & k_{53}        & k_{53}\,\ysh  & \frac{EI_y(2-\Phi_z)}{L(1+\Phi_z)} & 0                                  & 0      & 0       & -k_{53} & -k_{54} & k_{55} &        \\
%     0       & k_{62}        & 0             & -k_{62}\,\zsh & 0                                  & \frac{EI_z(2-\Phi_y)}{L(1+\Phi_y)} & 0      & -k_{62} & 0       & -k_{64} & 0      & k_{66} \\
%   \end{bmatrix}
%   \nonumber\\
% &\text{(to be used with the definitions from \autoref{eq:ShearDeformationParams} and \autoref{eq:Ksh})} \nonumber
% \end{align}
% \renewcommand*{\arraystretch}{1}
% 
% 
% 
% 
% % --------------------------------------------------------------------------------}
% % ---  
% % --------------------------------------------------------------------------------{
% \subsection{Timoshenko beam with shear center offsets - Beam along z}
% \label{sec:TimoshenkoBeamAlongZ}
% The equations presented in this section are similar to the ones presented in \autoref{sec:TimoshenkoBeamAlongX} but the coordinate system is rotated leading to different definitions. The notations are yet kept the same so the reader should not mix equations from the two sections.
% The coordinates $(x,y,z)$ of \autoref{sec:TimoshenkoBeamAlongZ} are now permuted to $(y,z,x)$.
% The $\Phi_x$, $\Phi_y$ and $K_{sh}$ are then defined as
% \begin{align}
% \Phi_x 
% &= \frac{12 EI_y}{k_x A G L^2}
% = \frac{12 EI_y}{{A_{s_x}} G L^2}
% = 24 ( 1+\nu) \frac{A}{A_{s_x}}\left(\frac{r_y}{L}\right)^2 \nonumber\\
% \Phi_y 
% &= \frac{12 EI_x}{k_y A G L^2}
% = \frac{12 EI_x}{{A_{s_y}} G L^2}
% = 24 ( 1+\nu) \frac{A}{A_{s_y}}\left(\frac{r_x}{L}\right)^2
%    \label{eq:ShearDeformationParamsAlongZ} \\
% K_\text{sh} &= \frac{12E}{L^3}\left[\frac{I_x \xsh^2}{1+\Phi_y}  + \frac{I_y \ysh^2}{1+\Phi_x} \right] \label{eq:KshAlongZ}
% \end{align}
% % with the following notation introduced by some authors: $\eta_x=\Phi_y/12$,  $\eta_y=\Phi_x/12$, $\rho_x=1/(1+\Phi_y)$, $\rho_y=1/(1+\Phi_x)$ (note the inversion between $x$ and $y$ in these definitions).
% % 
% The stiffness matrix for a Timoshenko beam element directed along the $z$ direction, with shear center offsets $\xsh$, $\ysh$ is given in \autoref{eq:KTimoshenkoOffsetsAlongZ}.
% % , where the following notation was introduced:
% % \begin{align}
% % \end{align}
% % and the notation $k_{ij}$ is used to denote the element of row $i$ and column $j$ of the matrix.
% 
% \renewcommand*{\arraystretch}{1.9}
% \begin{align}
% \setcounter{MaxMatrixCols}{20}
% % \m{K}&=\\
% &\text{Stiffness matrix for a Timoshenko beam element along $z$ with shear center offsets}\label{eq:KTimoshenkoOffsetsAlongZ}\\
% &\!\!  \begin{bmatrix}
%      \frac{12 E I_y}{L^3(1+\Phi_x)} &                                &              &                                    &                                    &                              &         &         &        &         &         &         \\
%      0                              & \frac{12 E I_x}{L^3(1+\Phi_y)} &              &                                    &                                    &  \text{sym.}                 &         &         &        &         &         &         \\
%      0                              & 0                              & \frac{EA}{L} &                                    &                                    &                              &         &         &        &         &         &         \\
%      0                              & -\frac{6EI_x}{L^2(1+\Phi_y)}   & 0            & \frac{EI_x(4+\Phi_y)}{L(1+\Phi_y)} &                                    &                              &         &         &        &         &         &         \\
%      \frac{6 EI_y}{L^2(1+\Phi_x)}   & 0                              & 0            & 0                                  & \frac{EI_y(4+\Phi_z)}{L(1+\Phi_x)} &                              &         &         &        &         &         &         \\
%      -k_{11}\,\ysh                  & k_{22}\,\xsh                   & 0            & k_{42}\,\xsh                       & -k_{51}\,\ysh                      & \frac{GI_z}{L} + K_\text{sh} &         &         &        &         &         &         \\
%      -k_{11}                        & 0                              & 0            & 0                                  & -k_{51}                            & k_{11}\,\ysh                 & k_{11}  &         &        &         &         &         \\
%      0                              & -k_{22}                        & 0            & -k_{42}                            & 0                                  & -k_{22}\,\xsh                & 0       & k_{22}  &        &         &         &         \\
%      0                              & 0                              & -k_{33}      & 0                                  & 0                                  & 0                            & 0       &  0      & k_{23} &         &         &         \\
%      0                              & k_{42}                         & 0            & \frac{EI_x(2-\Phi_y)}{L(1+\Phi_y)} & 0                                  & k_{42}\,\xsh                 & 0       & -k_{42} & 0      & k_{44}  &         &         \\
%      k_{51}                         & 0                              & 0            & 0                                  & \frac{EI_y(2-\Phi_x)}{L(1+\Phi_x)} & -k_{51}\,\ysh                & -k_{51} & 0       & 0      & 0       & k_{55}  &         \\
%      k_{11}\, \ysh                  & -k_{22}\,\xsh                  & 0            & -k_{42}\,\xsh                      & k_{51}\,\ysh                       & -k_{66}                      & k_{61}  & k_{62}  & 0      & -k_{64} & -k_{65} & k_{66}  \\
%   \end{bmatrix}
%   \nonumber
%       \\
% &\text{(to be used with the definitions from \autoref{eq:ShearDeformationParamsAlongZ} and \autoref{eq:KshAlongZ})} \nonumber
% \end{align}
% \renewcommand*{\arraystretch}{1}
% 
% 
% 
% \begin{lstlisting}
%   real*8              :: z=0.0d0, &             ! curved distance
%                          m=0.0d0, &             ! mass distribution
% 						 x_cg=0.0d0, &          ! Center of gravity related to centerline
% 						 y_cg=0.0d0, &          ! Center of gravity related to centerline
% 						 ri_x=0.0d0, &          ! Radius of gyration related to centeline
% 						 ri_y=0.0d0, &          ! Radius of gyration related to centeline
% 						 x_sh=0.0d0, &          ! Shear center related to centerline
% 						 y_sh=0.0d0, &          ! Shear center related to centerline
% 						 E=0.0d0, &             ! Modulus of elasticity
% 						 G=0.0d0, &             ! Shear modulus of elasticity
% 						 I_x=0.0d0, &           ! Areal moment of inertia related to 1st principal bending axes
% 						 I_y=0.0d0, &           ! Areal moment of inertia related to 2nd principal bending axes
% 						 I_p=0.0d0, &           ! Torsional stiffness constant
% 						 k_x=0.0d0, &           ! Shear parameter
% 						 k_y=0.0d0, &           ! Shear parameter
% 						 A=0.0d0, &             ! Cross sectional area
% 						 theta_z=0.0d0,&        ! Angle between 1st principal bending axe and aerodyn chord
% 						 x_e=0.0d0, &           ! Elastic axe center related to chord midpoint
% 						 y_e=0.0d0,  &           ! Elastic axe center related to chord midpoint
% \end{lstlisting}
% 
% 
% \begin{align}
%   \eta_x = \frac{E I_x}{k_y G A L^2} 
%   ,\qquad
%   \eta_y = \frac{E I_y}{k_x G A L^2} \\
%   \rho_x = \frac{1}{1+12\eta_x}
%   ,\qquad
%   \rho_y = \frac{1}{1+12\eta_y}\\
%   K_{11} = 
%   \begin{pmatrix}
%     \frac{12 E I_y \rho_y}{L^3}  & 0 & 0 & 0  \\ 
%   \end{pmatrix}
% \end{align}
% % ---
% % Code taken from Timoshenko.f90, ELSTIF
% % ---
% \begin{lstlisting}
% !    Calculates the element stiffness matrix.                          *
%       ETAX = EMOD*IX/(KY*G*A*L**2)
%       ETAY = EMOD*IY/(KX*G*A*L**2)
%       ROX  = 1/(1+12*ETAX)
%       ROY  = 1/(1+12*ETAY)
%       KK = 0.0D0
%       KK( 1, 1) =  12*E*IY*ROY/L**3
%       KK( 1, 5) =  6*E*IY*ROY/L**2
%       KK( 1, 6) = -KK(1,1)*YSH
%       KK( 1, 7) = -KK(1,1)
%       KK( 1,11) =  KK(1,5)
%       KK( 1,12) = -KK(1,6)
%       KK( 2, 2) =  12*E*IX*ROX/L**3
%       KK( 2, 4) = -6*E*IX*ROX/L**2
%       KK( 2, 6) =  KK(2,2)*XSH
%       KK( 2, 8) = -KK(2,2)
%       KK( 2,10) =  KK(2,4)
%       KK( 2,12) = -KK(2,6)
%       KK( 3, 3) =  E*A/L
%       KK( 3, 9) = -KK(3,3)
%       KK( 4, 4) =  4*E*IX*(1+3*ETAX)*ROX/L
%       KK( 4, 6) =  KK(2,4)*XSH
%       KK( 4, 8) = -KK(2,4)
%       KK( 4,10) =  2*E*IX*(1-6*ETAX)*ROX/L
%       KK( 4,12) = -KK(4,6)
%       KK( 5, 5) =  4*E*IY*(1+3*ETAY)*ROY/L
%       KK( 5, 6) = -KK(1,5)*YSH
%       KK( 5, 7) = -KK(1,5)
%       KK( 5,11) =  2*E*IY*(1-6*ETAY)*ROY/L
%       KK( 5,12) = -KK(5,6)
%       KK( 6, 6) =  G*IZ/L +12*E*(IX*XSH**2*ROX+IY*YSH**2*ROY)/L**3
%       KK( 6, 7) =  KK(1,12)
%       KK( 6, 8) =  KK(2,12)
%       KK( 6,10) = -KK(4,12)
%       KK( 6,11) = -KK(5,12)
%       KK( 6,12) = -KK(6,6)
%       KK( 7, 7) =  KK(1,1)
%       KK( 7,11) = -KK(1,5)
%       KK( 7,12) =  KK(1,6)
%       KK( 8, 8) =  KK(2,2)
%       KK( 8,10) = -KK(2,4)
%       KK( 8,12) =  KK(2,6)
%       KK( 9, 9) =  KK(3,3)
%       KK(10,10) =  KK(4,4)
%       KK(10,12) = -KK(4,6)
%       KK(11,11) =  KK(5,5)
%       KK(11,12) = -KK(5,6)
%       KK(12,12) =  KK(6,6)
% \end{lstlisting}



% --------------------------------------------------------------------------------}
% --- HAWC2 BeamDyn 
% --------------------------------------------------------------------------------{
% \subsection{HAWC2 - BeamDyn}




% --------------------------------------------------------------------------------}
% ---
% --------------------------------------------------------------------------------{
% \section{}


% \begin{itemize}\tightlist
%     \item 
% \end{itemize}
% ----------------------------------- TABLE --------------------------------------
% \begin{landscape}
% \begin{table}[!htb]\centering
%     \caption{}
%     \label{tab:Symbols}
% \begin{tabular}{cc}
%  \textbf{Symbol} & \textbf{Value}  \\
%   x   & 2\\
% \hline
%  \end{tabular}
% \end{table}
% \end{landscape}
% --------------------------------------------------------------------------------
% ---------------------------------- FIGURE --------------------------------------
% From script: Main*, from folder: , 30-Nov-2016
% \noindent\begin{figure}[!htbp]\centering%
%     \includegraphics[width=0.60\textwidth]{FigureExample}
%     \caption{Figure caption}
%     \label{fig:Contour}%
% \end{figure}
% --------------------------------------------------------------------------------
% ---------------------------- SKETCH --------------------------------------------
% \svgtex{FASTAirfoilSystem}{Sketch title}{0.6}{0.90}
% --------------------------------------------------------------------------------
% \python
% \begin{lstlisting}
% \end{lstlisting}




% --------------------------------------------------------------------------------}
% --- BIBLIO 
% --------------------------------------------------------------------------------{
\bibliographystyle{unsrt}
\bibliography{Bibliography}

\end{document}
