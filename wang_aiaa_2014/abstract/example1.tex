%&LaTeX
\subsection{Example 1: Static bending of a cantilever beam}

The first example is a benchmark problem for geometrically nonlinear
analysis of beams \cite{Simo1985,Xiao-Zhong:2012}. We calculate the static
deflection of a cantilever beam that is subjected at its free end to
a constant moment $M$.  The length of the beam $L$ is $10\,in$ and the cross-sectional stiffness 
matrix is given below:
\begin{equation}
    \label{StifE1}
    C^* = 10^3 \times \begin{bmatrix}
	1770 & 0    & 0    & 0    & 0    & 0   \\
	     & 1770 & 0    & 0    & 0    & 0   \\
	     &      & 1770 & 0    & 0    & 0   \\
	     &      &      & 8.16 & 0    & 0   \\
	     &      &      &      & 86.9 & 0   \\
	     &      &      &      &      & 215
\end{bmatrix}
\end{equation}
It is pointed out that the term with a asterisk denotes it is resolved in the material coordinate system. A sketch of this case can be found in Figure~\ref{E1Sketch}.
\begin{figure}
    \centering
    \includegraphics[width=3.0in]{\directory E1Sketch.eps}
    \caption{Sketch of a cantilever beam.}
    \label{E1Sketch}
\end{figure} 
The load applied at the tip is given by the following equation:
\begin{equation}
    \label{E1Load}
    M_2 = \lambda \bar{M}_2
\end{equation}
where $\bar{M}_2 = \pi \frac{EI_2}{L}$. The parameter $\lambda$ will vary between $0$ and $2$. In 
this case, the beam is discretized with two $5^{th}$ order elements. The deformations of the beam are shown in Figure~\ref{E1Deform}.
\begin{figure}
    \centering
    \includegraphics[width=5.0in]{\directory E1Deform.eps}
    \caption{Deformations of a cantilever beam under several constant bending moments.}
    \label{E1Deform}
\end{figure}
The calculated results are compared with the analytical solution, which can be found in Ref.\cite{Mayo-etal:2004} as
\begin{equation}
    \label{E1Analytical}
    u_1 = \rho sin \left( \frac{x_1}{\rho} \right) - x_1~~~~~u_3 = \rho \left(1-cos\left(\frac{x_1}{\rho}\right) \right)
\end{equation}
The results can be found in Table~\ref{E1u1} and \ref{E1u3}, respectively. Good agreement can be observed between these two sets of results.
\begin{table}[tbp]
\centering 
\caption{Axial displacement $u_1$ of a cantilever beam subject to a constant moment (in inches).}
\label{E1u1} 
	\begin{tabular}{| l | l | l | l |}
    	\hline
    	$\lambda$ & Analytical & BeamDyn  & \% Error \\ \hline
    	0.4       & -2.4317    & -2.4317  & 0.00       \\ \hline
    	0.8       & -7.6613    & -7.6613  & 0.00       \\ \hline
    	1.2       & -11.5591   & -11.5591 & 0.00       \\ \hline
    	1.6       & -11.8921   & -11.8921 & 0.00       \\ \hline
    	2.0       & -10.0000   & -10.0000  & 0.00       \\ \hline
    \end{tabular}
\end{table}

\begin{table}[tbp]
\centering 
\caption{Vertical displacement $u_3$ of a cantilever beam subject to a constant moment (in inches).}
\label{E1u3} 
	\begin{tabular}{| l | l | l | l |}
    	\hline
    	$\lambda$ & Analytical & BeamDyn & \% Error \\ \hline
    	0.4       & 5.4987     & 5.4987  & 0.00   \\ \hline
    	0.8       & 7.1978     & 7.1979  & 0.0013   \\ \hline
    	1.2       & 4.7986     & 4.7986  & 0.00   \\ \hline
    	1.6       & 1.3747     & 1.3747  & 0.00   \\ \hline
    	2.0       & 0.0000     & 0.0000  & 0.00   \\ \hline
    \end{tabular}
 \end{table}
 The rotation parameters at each node along beam axis $x_1$ obtained from BeamDyn are plotted in Figure~\ref{E1Rot} for $\lambda = 0.8$ and $\lambda = 2.0$, respectively. It is noted that the three-dimensional rotations are represented by Wiener-Milenkovi\'c parameter defined in the following equation:
 \begin{equation}
     \vec{p} = 4 tan\frac{\phi}{4} \bar{n}
     \label{WMParameter}
 \end{equation}
 where $\phi$ is the rotation angle and $\bar{n}$ is the unit vector of rotation axis. The singularity exists in the above definition can be removed by a rescaling operation, which can be observed in Figure~\ref{E1Rot}.
\begin{figure}
    \centering
    \includegraphics[width=3.0in]{\directory E1Rot.eps}
    \caption{Wiener-Milenkovi\'c rotation parameters along beam axis $x_1$ .}
    \label{E1Rot}
\end{figure}
Figure~\ref{E1Conv} shows the normalized error $\epsilon(u)$, where $u$ is the tip displacement (at $x=L$), as a function of the number of model nodes for the calculation with Dymore quadratic elements (QE) and a single Legendre spectral element (LSE), where
\begin{equation}
    \label{E1Error}
    \epsilon(u) = \left| \frac{u-u^a}{u^a} \right|
\end{equation}
and $u$ is the test solution and $u^a$ is the analytical solution. The parameter $\lambda$ is set to $1.0$ for this case. The Legendre spectral elements (with $p$-refinement) exhibit highly desirable exponential convergence to machine precision error.
\begin{figure}
    \centering
    \begin{tabular}{c}
    \subfloat[$u_1$]{\label{E1Conv:u1}\includegraphics[width=3.0 in]{\directory  E1Convu1.eps}} \qquad
\subfloat[$u_3$]{\label{E1Conv:u3}\includegraphics[width=3.0in]{\directory  E1Convu3.eps}}\\
\end{tabular}
\caption{Normalized error of the (a) $u_1$ and (b) $u_3$ displacements as a function of the total number of nodes}
\label{E1Conv}
\end{figure}

