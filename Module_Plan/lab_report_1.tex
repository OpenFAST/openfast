%%%%%%%%%%%%%%%%%%%%%%%%%%%%%%%%%%%%%%%%%
% University/School Laboratory Report
% LaTeX Template
% Version 3.0 (4/2/13)
%
% This template has been downloaded from:
% http://www.LaTeXTemplates.com
%
% Original author:
% Linux and Unix Users Group at Virginia Tech Wiki 
% (https://vtluug.org/wiki/Example_LaTeX_chem_lab_report)
%
% License:
% CC BY-NC-SA 3.0 (http://creativecommons.org/licenses/by-nc-sa/3.0/)
%
%%%%%%%%%%%%%%%%%%%%%%%%%%%%%%%%%%%%%%%%%

%----------------------------------------------------------------------------------------
%	PACKAGES AND DOCUMENT CONFIGURATIONS
%----------------------------------------------------------------------------------------

\documentclass{article}

\usepackage{mhchem} % Package for chemical equation typesetting
\usepackage{siunitx} % Provides the \SI{}{} command for typesetting SI units

\usepackage{graphicx} % Required for the inclusion of images

\setlength\parindent{0pt} % Removes all indentation from paragraphs

\renewcommand{\labelenumi}{\alph{enumi}.} % Make numbering in the enumerate environment by letter rather than number (e.g. section 6)

%\usepackage{times} % Uncomment to use the Times New Roman font

%----------------------------------------------------------------------------------------
%	DOCUMENT INFORMATION
%----------------------------------------------------------------------------------------

\title{Module BeamDyn} % Title

%\author{John \textsc{Smith}} % Author name

%\date{\today} % Date for the report

\begin{document}

\maketitle % Insert the title, author and date

%\begin{center}
%\begin{tabular}{l r}
%Date Performed: & January 1, 2012 \\ % Date the experiment was performed
%Partners: & James Smith \\ % Partner names
%& Mary Smith \\
%Instructor: & Professor Smith % Instructor/supervisor
%\end{tabular}
%\end{center}

% If you wish to include an abstract, uncomment the lines below
% \begin{abstract}
% Abstract text
% \end{abstract}

%----------------------------------------------------------------------------------------
%	SECTION 1
%----------------------------------------------------------------------------------------

\section*{Description}

This module represents blade in wind turbine systems.
The module is given the module name {\it ModuleName=BeamDyn} and the abbreviated name {\it ModName=BDyn}.
 
%----------------------------------------------------------------------------------------
%	SECTION 2
%----------------------------------------------------------------------------------------

\section*{Inputs, Outputs, States, and Parameters}

These are the nonlocal variables that must be defined in the module�s Registry.

\subsection*{Initialization Input}

\subsection*{Initialization Output}

\subsection*{Inputs $(\textbf{u})$}

$\omega_r = $ Angular velocity at the root of blade (rad/s)\\
$f = $ Distributed applied forces along beam axis (N/m) \\
$m = $ Distributed applied torque and bending moments (N)

\subsection*{Outputs $(\textbf{y})$}
$F_r = $ Stress Resultants at the root of blade (N) \\
$M_r = $ Moment resultants at the root of blade (N $\cdot$ m) \\
$U_{dis} = $ Displacement vector (m) \\
$U_{rot} = $ Rotation angle (rad) \\
$V_{lin} = $ Linear velocity vector (m/s) \\
$V_{ang} = $ Angular velocity vector (rad/s)

\subsection*{States}

\subsubsection*{Continuous States $(\textbf{x})$}
$u_{dis} = $ Displacement vector (m) \\
$\dot{u}_{dis} = $ Linear velocity vector (m/s) \\
$u_{rot} = $ Rotation vector\\
$\dot{u}_{rot} = $ Angular velocity vector

\subsubsection*{Discrete States $(\textbf{x}^d)$}

\subsubsection*{Constraint Staes $(\textbf{z})$}

\subsubsection*{Other States}

\subsection*{ Parameters $(\textbf{p})$}
$M_m = $ $6 \times 6$ Sectional mass matrix resolved in material basis \\
$C_m = $ $6 \times 6$ Sectional stiffness matrix resolved in material basis

\section*{Mathematical Formulation}


%----------------------------------------------------------------------------------------
%	BIBLIOGRAPHY
%----------------------------------------------------------------------------------------

\bibliographystyle{unsrt}

\bibliography{sample}

%----------------------------------------------------------------------------------------


\end{document}